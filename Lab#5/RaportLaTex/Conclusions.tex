\section*{Concluzie}
\phantomsection


In aceata lucrare de laborator am avut ocazia sa invat instrumentele de baza a limbajui java si implementarea acestora in IDE-ul Android Studio pentru dezvoltarea unei aplicatii. Am folosit API-urile java si android pentru simplificarea codului de baza, la fel am insusit cum lucreaza activity si layout la fel si definirea lor in android manifest, in general structura unei aplicatii android, resursele ei si modul de funcionare. Am mai avut ocazia sa folosesc instrumentele pentru mesaje text, dialog boxes si notificatiile cu sunet si vibratie care la rindul sau au un mod special de configurare prin builder. Am antrenat lucrul cu timerul folosind Chronometer si setindul manual al fiecare secunda de reactualizarii functii OnTick. La fel am avut nevoie de multe testari pentru depistarea bug-urilor ce a fost o experienta utila pentru viitoarea profesie in developmen.

\clearpage